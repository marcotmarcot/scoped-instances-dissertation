O sistema de módulos de Haskell objetiva a simplicidade e possui a notável
vantagem de ser fácil de aprender e usar.  Entretanto, instâncias de classes
de tipo em Haskell são sempre exportadas e importadas entre módulos.  Isso
quebra a uniformidade e simplicidade do sistema de módulos e introduz problemas
práticos.  Instâncias criadas em módulos diferentes podem conflitar uma com a
outra e podem fazer com que seja impossível importar dois módulos que contenham
definições de uma mesma instância se essa instância for utilizada.  Isso faz com que seja
muito incoveniente a definição de duas instâncias diferentes da mesma classe de
tipos para o mesmo tipo em diferentes módulos de um mesmo programa.  A definição
de instâncias em módulos onde nem o tipo nem a classe de tipos são definidos se tornou uma má prática, e essas
instâncias foram chamadas de instâncias órfãs.  Somente esse tipo de instância
pode causar conflitos já que, se instâncias forem definidas apenas no
mesmo módulo do tipo ou da classe de tipos, só poderá existir uma instância para
cada par de classe e tipo.

Nessa dissertação
nós apresentamos e discutimos uma solução para esses problemas que simplesmente
permite que haja controle sobre a importação e exportação de instâncias entre
módulos, através de uma pequena alteração na linguagem.  A solução é apresentada
em duas versões.  A versão final, mais consistente, não é compatível com
Haskell, isto é, programas que funcionam em Haskell podem deixar de funcionar
com essa alteração.  Já a versão intermediária traz os benefícios da proposta
mesmo sendo compatível com Haskell, mas é um pouco menos consistente.
Para evitar que o programador precise escrever nomes de instâncias muito longos
nas listas de controle de importação e exportação de módulos, propomos
outra pequena alteração na linguagem, que torna possível dar nomes mais curtos a
instâncias.

Também mostramos como a especificação formal do sistema de módulos precisa
ser adaptada para lidar com nossa proposta.  Como a especificação formal não
tratava instâncias, primeiro adaptamos essa especificação
para tratar instâncias e, em seguida, mostramos como nossa proposta é
especificada formalmente.

\keywords{Instâncias de classes de tipo, Módulos, Haskell}
