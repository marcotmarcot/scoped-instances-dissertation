\documentclass[msc]{ppgccufmg}

\usepackage[english]{babel}
\usepackage[utf8x]{inputenc}
\usepackage[T1]{fontenc}
\usepackage{type1ec}
\usepackage[square]{natbib}
\usepackage[a4paper,
  bookmarks=true,
  bookmarksnumbered=true,
  linktocpage,
  colorlinks,
  citecolor=black,
  urlcolor=black,
  linkcolor=black,
  filecolor=black,
  ]{hyperref}
\begin{document}

\ppgccufmg{
  title={Controlling the scope of instances in Haskell},
  authorrev={Gontijo, Marco},
  university={Federal University of Minas Gerais},
  course={Computer Science},
  portuguesetitle={Controlando o escopo de instâncias em Haskell},
  portugueseuniversity={Universidade Federal de Minas Gerais},
  portuguesecourse={Ciência da Computação},
  address={Belo Horizonte},
  date={2012-11},
  keywords={Type class instances, Modules, Haskell},
  advisor={Carlos Camarão},
  abstract=[brazil]{Resumo}{resumo},
  abstract={Abstract}{abstract},
  dedication={dedicatoria},
  ack={agradecimentos},
  epigraphtext={Eu quase que nada não sei.  Mas desconfio de muita coisa.}{Riobaldo}
}

\chapter{Introduction}
Modern programming languages are using more flexible type systems in order to accept a larger set of programs.
The goal of these type systems is to reject less programs that would work correctly but that would not be accepted by more restrictive type systems.
One of the techniques to achieve this flexibility is to promote code reuse by supporting polymorphism, which allows the same code to be used with distinct data types.
There are different approaches to polymorphism, one of them being ad-hoc, or constrained, polymorphism \citep{wadler}, which supports code that use overloaded names (or symbols) and reuse of such code for all data types for which a definition of the overloaded names have been given.

In C++ this kind of polymorphism is achieved by means of overloading function names \citep[section 7.4]{stroustrup}.
From the compiler perspective, two functions with the same name but with different types as the parameter are considered two different functions.
When there are more than one option of function to call for a specific type, or for a polymorphic symbol, like numerals, the compiler applies a set of rules to decide between them.
In order to avoid the need to understand the context to define what of the functions with the same name were called, ``return types are not considered in overloading resolution'' \citep[section 7.4.1]{stroustrup}.

\section{Type classes}

Haskell is a programming language that is nowadays used in academic research specially to study and experiment with topics related to type systems and type inference, and is also being used in commercial applications\footnote{\url{http://industry.haskell.org/}}.
Type classes are a language mechanism that was introduced in Haskell for supporting ad-hoc polymorphism \citep{tch}.
A type class specifies a set of overloaded names together with type annotations for them.
For instance, the names defined for class \texttt{Eq}, from the Haskell Prelude \citep{report} are given in Figure \ref{Eq}.
As can be seen in this Figure, the definition of the class can contain a standard implementation of some methods, and this implementation can be based on other methods of the same class.

\begin{figure}
\caption{Definition of Class \texttt{Eq} from the Haskell Prelude.\label{Eq}}
\begin{tabular}{|p{\textwidth}|}
\hline
\begin{verbatim}
class Eq a where
  (==), (/=) :: a -> a -> Bool

  x == y = not (x /= y)
  x /= y = not (x == y)
\end{verbatim}
\\
\hline
\end{tabular}
\end{figure}

An implementation of a type class for a data type, called an instance of the type class, provides definitions for all overloaded names of that type class.
As an example, a possible definition of an instance of \texttt{Eq} for \texttt{Bool} can be seen at Figure \ref{EqBool}.
The definition of the data type \texttt{Bool}, from the Haskell Prelude \citep{report} can be seen at Figure \ref{Bool}.

\begin{figure}
\caption{\texttt{Bool} data type as defined in the Haskell Prelude.\label{Bool}}
\begin{tabular}{|p{\textwidth}|}
\hline
\begin{verbatim}
data Bool = False | True
\end{verbatim}
\\
\hline
\end{tabular}
\end{figure}

\begin{figure}
\caption{Instance of type class \texttt{Eq} for type \texttt{Bool}.\label{EqBool}}
\begin{tabular}{|p{\textwidth}|}
\hline
\begin{verbatim}
instance Eq Bool where
  True == True = True
  False == False = True
  _ == _ = False
\end{verbatim}
\\
\hline
\end{tabular}
\end{figure}

\section{Modules}

A module system of a programming language is intended to provide
support for a modular construction of software systems.  In some
languages the module system provides a type-safe abstraction
mechanism, where definitions can be parameterized so that
modules can be instantiated for different kinds of
entities. This is the case for example of Standard ML \citep{sml}.

A module system can also merely allow a program to
be divided into parts that can be compiled separately, such as in Java \citep{java} or Scala \citep{scala}.
In some other
languages, the module system provides a mechanism to control the
visibility of globally defined names, either to hide
implementation-specific details or to access parts that would
otherwise be out of scope. This is the case for example of Haskell
\citep[chapter~5]{report}.

The Haskell module system aims for simplicity \citep[section~8.2]{history}
and has the notable advantage of being easy to learn and use.
It provides control over exported entities through export lists, which is used to construct abstract data types.
It also provides control over imported entities through import lists, which can avoid conflicts of names.
Another way to avoid name conflicts is to import modules qualified.
In this case, each usage of a name from the imported module must be prefixed by the module name.
As module names can be long, they can be locally renamed to make the prefix less verbose.

\section{Contributions of this dissertation}

However, this simplicity is partly hindered by the special treatment given to the scope of instances, for which there are not control on exportation and importation.
As defined in the Modules chapter of the Haskell 2010 Report \citep[section~5.4]{report}, a type class ``instance declaration is in scope if and only if a chain of \texttt{import} declarations leads to the module containing the instance declaration''.

Because of this, it is not possible for a module to import two modules that defines the same instance, that is, an instance of the same type class to the same data type, if the importing module, or any module that imports it, use the instance.  This happens if the both if the definitions are different or the same on the different modules.  This is a
serious
restriction. The aim is, as in all type system restrictions, to
prevent the programmer from making mistakes.  However, even though
this design decision protects the programmer from incurring in some
mistakes, it can also disallow reasonable and correct code.
Furthermore, a lot of instances generally become part of the scope of
modules without ever being used.  This puts a burden on compiler
writers, which have to consider smart ways of controlling the size of
the scope of modules.

In this dissertation we propose an extension to Haskell that
allows programmers to control when to export and import instances.
This makes it possible to create instances local to a module or
visible only in a subset of modules of a program, and removes problems
brought by importation of modules that contain definitions of
instances for the same type.

\section{Outline of this work}
Chapter \ref{Background} illustrates how the abscence of control of the visibility of instances makes it hard or impossible to use instances for a certain type with a special purpose, in section \ref{Special-purpose-instances}; and describes orphan instances, an instance that is not defined neither in the module that defined the type class nor in the module that defined the data type, in section \ref{Orphan-instances}.

In Chapter \ref{solution} we present our
proposal, with two possible alternatives: the final, in section \ref{final}; and the intermediate, in section \ref{intermediate}.
This chapter also presents a complementary proposal that give names to instances, on section \ref{Instance-names}.
Some consequences of the control over the scope of instances are disscussed on section \ref{Instance-scope} and the implementation of the proposal is detailed on section \ref{Implementation}.
A discussion about problems that can occur by the adoption of our proposal, and possible solutions to them is presented on section \ref{problems}.

Chapter \ref{formal} describes one way of extending a published
formalization of Haskell's module system \citep{formal} to handle our proposal.
As instances were not modeled on the original formalization, they are treated at first, and then our proposal was included.

Chapter \ref{related} describes related work and chapter \ref{conclusion} concludes the dissertation.

\chapter{Background}
\label{Background}

\section{Defining special purpose instances}
\label{Special-purpose-instances}

As instances are always exported and imported, all instances defined in the
program will be available at the topmost module of the program, that is, the
\texttt{Main} module.  If two instances of the same type class for the same
data type are defined in different parts of the program, some modules of the
program will have both of them available.  In the best case, only the
\texttt{Main} module will have them available, but the number of modules with
the two instances available can be much bigger.  In these modules any use of
one instance will result in a
compilation error.  So, in this scenario it is impossible to use an overloaded
function for a given type, even if the programmer knows which instance desired.  Because of this restriction, although it is possible
to use more than one instance of a type class for a type in a program, it is
very inconvenient, since the usage of an overloaded function for a type would be
lost in some parts of the program.  Also, it is not very useful, since each polymorphic
  functions that use this instance will have to be instantiated in the module it was defined, that is, they can not be exported to another module as a polymorphic function,
  to avoid instance conflict in an upper module on the import tree.

For example, in the \texttt{Main} module defined
in Figure \ref{main} the overloaded function \texttt{g} cannot be used.  Either
\texttt{g1}, defined at module \texttt{I1} in Figure \ref{I1}; or \texttt{g2},
defined at module \texttt{I2} in Figure \ref{I2} would have to be used.  The
instantiated version of each polymorphic function that uses one of the overloaded
definitions from the type class would have to be instantiated in the same module
that defines the instance.  Also, it will not be possible to use any overloaded
function defined at modules unknown to \texttt{I1} or \texttt{I2}.  These are significant disavantages for the use of type classes.

\begin{figure}
\caption{Module T.\label{T}}
\begin{tabular}{|p{\textwidth}|}
\hline
\begin{verbatim}
module T where

class T a where
  t :: a

g :: T a => (a, a)
g = (t, t)
\end{verbatim}
\\
\hline
\end{tabular}
\end{figure}

\begin{figure}
\caption{Module D.\label{D}}
\begin{tabular}{|p{\textwidth}|}
\hline
\begin{verbatim}
module D where

data D = D
\end{verbatim}
\\
\hline
\end{tabular}
\end{figure}

\begin{figure}
\caption{Module I1.\label{I1}}
\begin{tabular}{|p{\textwidth}|}
\hline
\begin{verbatim}
module I1 where

import T
import D

instance T D where
  t = undefined

g1 :: (D, D)
g1 = g

i1 :: a
i1 = undefined
\end{verbatim}
\\
\hline
\end{tabular}
\end{figure}

\begin{figure}
\caption{Module I2.\label{I2}}
\begin{tabular}{|p{\textwidth}|}
\hline
\begin{verbatim}
module I2 where

import T
import D

instance T D where
  t = undefined

g2 :: (D, D)
g2 = g

i2 :: a
i2 = undefined
\end{verbatim}
\\
\hline
\end{tabular}
\end{figure}

\begin{figure}
\caption{Main module of the example of orphan instances.\label{main}}
\begin{tabular}{|p{\textwidth}|}
\hline
\begin{verbatim}
import I1
import I2

f :: a -> a -> a
f = undefined

h :: a
h = f i1 i2
\end{verbatim}
\\
\hline
\end{tabular}
\end{figure}

Due to the inconvenience of defining and using more than one instance for a given type,
the programmer will not be able, for example, to sort values of a given type by using two
different techniques, applying an overloaded function \texttt{sort}.  More specifically, a
programmer can not use case-sensitive ordering to sort a list of strings in a part of a
program and case-insensitive ordering in another.

A general way to work around these problems is to create a new encapsulated data type, using \texttt{newtype}, and define a different instance for it.
The example in Figure \ref{newtype} illustrates this solution.  This works, but
it is verbose and not efficient.  In other words, it is ``too
clunky''\footnote{In Lennart Augustsson's
  words. \url{http://lukepalmer.wordpress.com/2009/01/25/a-world-without-orphans/\#comment-609}}.  It is a simple solution that can be considered good enough
for this problem, but it does not address the problem of the pollution of the global
scope.

\begin{figure}
\caption{Example of the usage of \texttt{newtype} to create a new
  instance.\label{newtype}}
\begin{tabular}{|p{\textwidth}|}
\hline
\begin{verbatim}
import Data.List

newtype IChar = IChar Char

unbox :: IChar -> Char
unbox (IChar c) = c

instance Eq IChar where
  (IChar c1) == (IChar c2) = iEq c1 c2

instance Ord IChar where
  compare (IChar c1) (IChar c2) = iCmp c1 c2

iSort :: [String] -> [String]
iSort = map (map unbox) . sort . map (map IChar)
\end{verbatim}
\\
\hline
\end{tabular}
\end{figure}

A less verbose solution exists, with the definition and use of
functions that include additional parameters instead of methods of
type classes. For example, module \texttt{Data.List} defines function
\texttt{sortBy :: (a -> a -> Ordering) -> [a] -> [a]}, which sorts the
list passed as the second parameter using the comparison function
given by the first parameter.  This is a simple and useful solution to
each specific problem such as this one, but it does not scale well.  To apply the same
idea generally, for all functions that use a type class method a
similar function having an additional parameter used instead of the
type class method would be necessary. This is not reasonable since it would add parameters in lots of cases, making the code
more complicated.  Also, it goes against the idea of making code
simpler and more reusable by means of overloading.

\section{Orphan instances}
\label{Orphan-instances}

The global visibility of type class instances allows the creation of so-called {\em
  orphan instances\/}.  Orphan instances are instances defined in a
module that contains neither the definition of the data type nor the
definition of the type class.  When an instance is defined in a module
where the data type or the type class is defined, it is guaranteed
that there will not exist more than one instance for each type class
and data type.  Orphan instances thus enable the
creation of distinct instances of a type class for the same data type.

They are specially troublesome when a module defines other functions that are
not related with the instance.  For example, if we have a module \texttt{T}
(Figure \ref{T}) that defines a type class \texttt{T}, a module \texttt{D} (Figure \ref{D}) that defines a data
type \texttt{D}, and two modules \texttt{I1} (Figure \ref{I1}) and
\texttt{I2} (Figure \ref{I2}) that define
instances of \texttt{T} for \texttt{D}, we would not be able to import both
\texttt{I1} and \texttt{I2} in the same
module, if this module uses \texttt{f}, or a function overloaded on
  class \texttt{T}.

In the example we are more interested in types and visibility control
by the module system than in the body of the presented functions.
Therefore, we are using function \texttt{undefined}, but the problem
remains the same if there was a relevant function body.

Instances defined in \texttt{I1} and \texttt{I2} are orphan instances.
The problem gets worse when there is a need to use, in the same
module, functions that are not related to instances, like \texttt{i1}
and \texttt{i2}.  It is not possible to use \texttt{i1} and \texttt{i2} on
the same program without modifying \texttt{I1} or \texttt{I2}. Even if \texttt{i1}
and \texttt{i2} are used in different modules, the \texttt{Main} module will
have to import both of them or a module which imports them.  If the
\texttt{Main} module, or some other module where both instances are availabe,
uses \texttt{f}, or a function overloaded on \texttt{T}, it will not be possible
to import \texttt{I1} and \texttt{I2} on the same program.  Modifying
\texttt{I1} or \texttt{I2} is not always possible in practice because they
may be part of a third-party library.

It is worth noticing that these are not only potential problems. They
happen in real world uses of the language.  For example, the Monad
instance of Either is defined in both packages \texttt{mtl} and
\texttt{transformers}\footnote{This example is on the wiki page at
  \url{http://www.haskell.org/haskellwiki/Orphan_instance} .}.  There
are examples where orphan instances would be desirable, involving
pretty printing and JSON\footnote{This example was presented by
  Lennart Augustsson in
  \url{http://lukepalmer.wordpress.com/2009/01/25/a-world-without-orphans/\#comment-601}
  .}.  Also, a situation has been reported where instances created
with Template Haskell could not be defined in the same module of the
data type or type class\footnote{Johan Tibell gives a detailed
  description of the situation in an e-mail at
  \url{http://www.haskell.org/pipermail/glasgow-haskell-users/2010-August/019052.html}
  .}.

\chapter{Solution}
\label{solution}
%In our view, this is a serious problem in the Haskell module system, which was
%designed with simplicity, rather than completeness, in mind.
%In this work we propose a solution to this problem.

We propose that instances should be exportable and importable.  It is a natural, simple proposal that has already
been mentioned\footnote{By Yitzchak Gale on Stack Overflow
  \url{http://stackoverflow.com/questions/3079537/orphaned-instances-in-haskell/3079748\#3079748}
  .}, but this work provides a detailed description and discussion,
including required changes in the language definition.

The proposal eliminates orphan
instances: the fact that a module defines an instance without
defining the related data type or type class does not cause any bad
consequence, since the programmer can choose which instance to use
by importing one module instead of another, and it can still use
functions defined in both modules, by hiding instances in an import
clause.  The \texttt{sortBy} problem is also solved, because
programmers can change the instance of a type class for a data type in
the context of a module, making it possible to call \texttt{sort} with
the desired instance defined in this module.

We examine two alternative syntaxes for the new language feature: a
backwards compatible one, referred to as \textbf{intermediate} --- but
not very uniform --- and a backwards incompatible one, called
\textbf{final}, which is more uniform.

If adopted, these alternative proposals should
preferably be enabled by compilers by the use of a compilation flag.  There
should exist then a different flag for each proposal.

In both cases, \texttt{export} and \texttt{import} clauses used in
The Haskell 2010 Report \citep[sections 5.2 and 5.3]{report} are
changed to have a new option, with the header of
an instance declaration \citep[section~4.3.2]{report}: \texttt{instance
  [scontext =>] qtycls}.  The option identifies whether an instance should be
exported, imported or hidden.  \texttt{import} and \texttt{export} clauses with the new option are defined as in
Figures \ref{export} and \ref{import}.

\begin{figure}
\caption{New syntax for the export clause.\label{export}}
\begin{tabular}{|l l l l|}
\hline
export & $\to$ & qvar &\\
& $|$ & qtycons [(..)$|$(cname$_1$, ..., cname$_n$)] & $(n \geq 0)$\\
& $|$ & qtycls [(..)$|$(var$_1$, ..., var$_n$)] & $(n \geq 0)$\\
& $|$ & \texttt{module} modid &\\
& $|$ & \texttt{instance} [scontext $=>$] qtycls &\\
\hline
\end{tabular}
\end{figure}

\begin{figure}
\caption{New syntax for the import clause.\label{import}}
\begin{tabular}{|l l l l|}
\hline
import & $\to$ & var &\\
& $|$ & tycon [(..)$|$(cname$_1$, ..., cname$_n$)] & $(n \geq 0)$\\
& $|$ & tycls [(..)$|$(var$_1$, ..., var$_n$)] & $(n \geq 0)$\\
& $|$ & \texttt{instance} [scontext $=>$] qtycls &\\
\hline
\end{tabular}
\end{figure}

\section{Final alternative}
\label{final}
In the final alternative instances are imported and exported just as other entities in Haskell.  There are ten distinct
cases where import clauses are affected by the proposal, presented below by
considering the example of at figure  module
\texttt{I1} presented previously at Figure \ref{I1}, similarly to
\cite[section~5.3.4]{report}:
\begin{enumerate}
\item \texttt{import I1} imports everything from module \texttt{I1},
  including instances, as occurs currently in Haskell;
\item \texttt{import I1 ()} imports nothing, as occurs if this line
  is commented or absent;
\item \texttt{import I1 (instance T D)} imports only the instance, which
  would be the same as \texttt{import I1 ()} in Haskell 2010;
\item \texttt{import I1 hiding (instance T D)} imports everything but
  the instance;
\item \texttt{import I1 (i1)} imports only \texttt{i1}, and not the
  instance.
\item \texttt{import qualified I1} makes all definitions accessible in a
  qualified manner, but not the instances;
\item \texttt{import qualified I1 ()} imports nothing, as occurs if this line is
  commented or absent, as in item (2);
\item \texttt{import qualified I1 (instance T D)} imports only the instance,
  as in item (3);
\item \texttt{import qualified I1 hiding (instance T D)} makes all definitions
  accessible in a qualified manner, but not any of the instances, including the
  one mentioned, as in item (6);
\item \texttt{import qualified I1 (i1)} makes only i1 accessible in a qualified
  manner.
\end{enumerate}

The only instance defined in \texttt{I1} is
\texttt{instance T D}.  If there were other instances to be imported, they should be also included
where \texttt{instance T D} is listed.

Usually, qualified imports are done as in item (6), because there's not much advantage of restricting the imported entities as in item (10), if you must specify the module on each usage.
Items (7), (8) and (9) are not very used because they have the same semantics of other items, which are more simple.
They are (2), (3) and (6) respectively.
In Haskell 2010 instances are imported even if the import is qualified.
This have the undesirable property that the inclusion of a qualified import in a module can create a compilation error of conflicting instances.
In the final alternative of our proposal, the inclusion of a qualified import does not import the instances.
If they should be imported, they must be mentioned on the import list.
This provides the property that the inclusion of a qualified import in a module will not bring any compilation errors.
Also, it is more consistent with the way qualified imports work with other entities: they do not affect the general scope of the importing module, they only create a way to access the entities from the imported module.
If instances were to be imported in qualified imports, as they are in Haskell 2010, the general scope of the module would be affected.

Similarly, there are four cases of export clauses affected by the proposal:
\begin{enumerate}
\item[11.] \texttt{module I1 where} exports everything in \texttt{I1}, including the
instance, as occurs currently in Haskell;
\item[12.] \texttt{module I1 () where} exports nothing, not even the
  instance;
\item[13.] \texttt{module I1 (instance T D) where} exports only the
  instance, such as \texttt{module I1 () where} in Haskell 2010;
\item[14.] \texttt{module I1 (i1) where} exports only \texttt{i1}, and not the
instance.
\end{enumerate}

This syntax is not backwards compatible because the behavior of a program that
contains a clause given in (2), (5), (7), (10), (12) or (14) is correct in Haskell 2010, but has a
different meaning than the one we are proposing.  In Haskell 2010, the instance
is imported or exported but in our proposal, it is not.  In our view this language extension should
be incorporated in the language in a second step, after the adoption of the intermediate alternative, described next.

\section{Intermediate alternative}
\label{intermediate}
The intermediate alternative differs from to the final alternative, just so as to be backwards compatible.  In items (2), (5), (7), (10), (12) and (14) instances are
imported or exported.  The only way to avoid an instance from being imported
is by using keyword \texttt{hiding} in an import list.  There is no way to
avoid an instance from being exported.
In the intermediate alternative, mentions of instances in the import and export lists are not considered if they are not on the hiding list.
Therefore, (13) is valid and has the same effect as (12), and the same goes for items (3) and (8).

The semantics of the intermediate alternative can be expressed using the syntax
of the final
alternative.  The interpretation of the examples that have their meanings changed
are rewritten in Figure \ref{tab}.  As the intermediate alternative has a syntax
that is backwards
compatible with Haskell 2010, Figure \ref{tab} also shows how Haskell 2010
constructs are mapped to the syntax of the final alternative.

\begin{table}
\caption{The semantics translation from the intermediate syntax to the
  final.\label{tab}}
\begin{tabular}{|r|l|l|}
\hline
& \textbf{Intermediate (or Haskell 2010)} & \textbf{Final} \\
\hline
2 & \texttt{import I1 ()} & \texttt{import I1 (instance T D)}\\
5 & \texttt{import I1 (i1)} & \texttt{import I1 (i1, instance T D)}\\
7 & \texttt{import qualified I1 ()} & \texttt{import qualified I1 (instance T D)}\\
10 & \texttt{import qualified I1 (i1)} & \texttt{import qualified I1 (i1, instance T D)}\\
12 & \texttt{module I1 () where} & \texttt{module I1 (instance T D) where}\\
14 & \texttt{module I1 (i1) where} & \texttt{module I1 (i1, instance T D)
    where}\\
\hline
\end{tabular}
\end{table}

The intermediate alternative has the same advantages of the final alternative, but it is less
uniform and should be used temporarily while programs are adapted to use
the syntax of the final alternative.  During this period, using constructions (2), (5), (7), (10), (12) and (14)
should be considered as bad programming practice.  These should be gradually
replaced by their final version, as shown in Figure \ref{tab}.  The final
version is also a valid intermediate syntax program, with the same meaning.

After this period, when the syntax of the final alternative becomes used, the use of these
constructions --- that is, (2), (5), (7), (10), (12) and (14) --- should be acceptable, but
they will have the semantics defined here, and not the old semantics.

New languages claims to justify their
existence fall under three categories \citep[p.~1]{claims}: ``novel features, incremental improvement on
existing features, and desirable language properties''.  This work presents
a language extension, which also needs a justification.  Our proposal as a whole can be seen as incremental improvement
on existing features, because it is not creating something new, but it is
improving the use of something that already exists.  The difference between the
intermediate and the final variations brings desirable language properties, which is
uniform behavior for similar constructs.

\section{Instance names}
\label{Instance-names}
A complementary syntax that could be added as an extension, and
enabled by a compiler using yet another compilation flag, is the
attribution of names to instances.  The motivation for this is that
sometimes instance contexts and types that identify instances can be
quite long and complex.  For example, \texttt{instance (Eq a, Eq b, Eq
  c, Eq d, Eq e, Eq f, Eq g, Eq h, Eq i, Eq j, Eq k,\\Eq l, Eq m, Eq n,
  Eq o) => Eq (a, b, c, d, e, f, g, h, i, j, k, l,\\m, n, o)} is
defined in the Haskell Prelude.  It would be better to create a name for
this instance, like EqTuple15, and use this name in import and export
lists.

This, as the rest of the proposal, would syntactically affect only the module
system.  The programmer will be able to create a synonym to refer to the
instance in export and export lists.  The idea of creating a synonym is similar
to the \texttt{type} construction in Haskell.

Naming of instances can be done using a top-level declaration like in, for
example, \texttt{inst Inst1 = instance
  T D}.  After an instance synonym is declared, it would be possible to use the
introduced name on import and export lists.  For instance: \texttt{import
  I1 hiding (Inst1)}.

Although it has a similar name, the Named Instances proposal
\citep{named} is very different from ours, because it requires more
significant changes to the language.  More details about how our work
is related to others is present on Chapter \ref{related}.

\section{Instance scope}
\label{Instance-scope}
Although the control of the visibility of instances allows control of
which entities are necessary and should actually be in the scope of
modules, there are subtle and somewhat unfortunate consequences of
such control. The most notable one is that a type annotation may cause
the semantics of the annotated construct to be changed. 

To see this, consider the example in Figure \ref{I1-2}, and two cases.
In the first, there is no type annotation of the type of function
\texttt{i1}, or there is an annotation, like \texttt{i1 :: T a => a},
that does not instantiate the constraint on \texttt{T}.  In the other
case, the type of \texttt{i1} is annotated so as to instantiate the
constraint on \texttt{T}, as for example \texttt{i1 :: D}.

\begin{figure}
\caption{Second version of module I1, using the proposed extension.\label{I1-2}}
\begin{tabular}{|p{\textwidth}|}
\hline
\begin{verbatim}
module I1 where

import T
import D

inst Inst1 = instance T D

instance T D where
  t = undefined

-- i1 :: D
i1 = t
\end{verbatim}
\\
\hline
\end{tabular}
\end{figure}

If the main module (Figure \ref{main-2})
did not import module \texttt{I2}, it would not be able to instantiate function
\texttt{i1} to \texttt{D}.  In the example presented, it will instantiate the
function to \texttt{D}, but using the instance defined in \texttt{I2}.
Therefore, the writer of module \texttt{I1} should notice that the instance
defined there will not necessarily be visible in the imported module and, when
there is an instance visible, it will not necessarily be the one defined in
module \texttt{I1}.

\begin{figure}
\caption{Second version of the main module, using the proposed
  extension.\label{main-2}}
\begin{tabular}{|p{\textwidth}|}
\hline
\begin{verbatim}
import I1 hiding (Inst1)
import I2

f :: D -> b -> b
f = undefined

g :: a
g = f i1 i2
\end{verbatim}
\\
\hline
\end{tabular}
\end{figure}

Also, the programmer should be aware that if the type annotation is
included, by uncommenting the line in module \texttt{I1}, the instance
defined in module \texttt{I1} will be used, even though it is not
visible in module main.  As already stated, if the line is commented,
the instance defined in \texttt{I2} will be used.

\section{Implementation}
\label{Implementation}
Usually, a compiler keeps a list of available instances while building a
module.  This list is used to check if an instance is available when inferring and
checking types, and to choose which instance to use when generating code.
Currently, instance visibility can not be controlled, so instances are only
included in this list, and there is no need for compilers to remove any element
of this list.  The implementation of our proposal will require removing
elements from this list while importing and exporting definitions from a module.

Our proposal aims to be simple and require as few changes to the
language as possible. This is noticed when the implementation details
are made clear: it is only a matter of filtering imported or exported
instances when requested.

\section{Problems and Solutions}
\label{problems}
Like most changes to an established language, this proposal has
its pros and cons.  Considering that ``a new language feature is only
justifiable if it results in a simplification or unification of the
original language design, or if the extra expressiveness is truly
useful in practice'' \citep[p.~1]{tc}, we judge that this language
feature is justifiable because the extra expressiveness added to
Haskell is truly useful in practice.  The main force that pushes
research in this field is the desire to have more well typed programs
\citep[p.~3]{pierce}, and this is our motivation.

On the other hand, there are reasons why this proposal was not included in the
language in the first place.  
It may be argued that changing the definition of
an instance of a class to a type in a program makes it harder to understand
what the code means.  
This is only a problem if the changes made to the
definitions are not intuitive in the program context, and this is not a problem
of the language extension per se, but of a possible use of it.  In Haskell,
it is already possible to break intuitivity with expressions like \texttt{let 1 + 1 = 3
in 1 + 1}, which overloads a function in local scope, without properly changing
the related type class or its instances.  So, this is not going to be the only
case in the language where basic constructions can have their meaning changed.

Changes to instance definitions can cause potentially unexpected
things to happen. Consider the following example. Suppose that a value
of type \texttt{Set} is internally represented by an ordered structure
of its elements, and that is why common operations, like insert,
requires the type to be an instance of \texttt{Ord}.  If a value of
type \texttt{Set Char} is defined in a module where the visible
instance of \texttt{Ord Char} is the default, and then used in a
module where a case-insensitive instance is visible, the search
operation can give perhaps unexpected results.

In module \texttt{Definition} (Figure \ref{definition}) \texttt{'a'}
will be inserted after \texttt{'B'}, since in case-sensitive order it
comes later.  Suppose \texttt{iCmp} is the comparison function
for case-insensitive Char.  The call of \texttt{member} on the main module
(Figure
\ref{main-set}) will search for \texttt{'a'} before \texttt{'B'},
because that is the case-insensitive order, and it will not find it,
returning \texttt{False}.  This is arguably not a good thing, but it is caused
by a misuse of a
feature. Dealing with it requires programmers to be careful when using
different instances of a type class for the same type in programs.

\begin{figure}
\caption{Module Definition, used in the example of unexpected behavior that
  arises from misuse of local instances.\label{definition}}
\begin{tabular}{|p{\textwidth}|}
\hline
\begin{verbatim}
module Definition where

import Data.Set

s :: Set Char
s = insert 'a' $ insert 'B' empty
\end{verbatim}
\\
\hline
\end{tabular}
\end{figure}

\begin{figure}
\caption{Main module of the example of unexpected behavior that arises from
  misuse of local instances.\label{main-set}}
\begin{tabular}{|p{\textwidth}|}
\hline
\begin{verbatim}
import Definition hiding (instance Ord Char)
import Prelude hiding (instance Ord Char)

instance Ord Char where
  compare = iCmp

m :: Bool
m = member 'a' s
\end{verbatim}
\vspace{-0.7cm}\\
\hline
\end{tabular}
\end{figure}

Another issue is related to the fact that the semantics of a function
may change because of the inclusion or not of a type
signature.\footnote{Simon Peyton-Jones states that type annotations
  should not change the result of a function in this e-mail:
  \url{http://www.haskell.org/pipermail/haskell/2001-May/007111.html}
  .} Although this is in general undesirable, in this case, when a
type is annotated with a less general type, an instance is being
chosen.  The instance to be used should be the one available in the
module where it was chosen, and not in the module where the exported
function is used. In the example with the module \texttt{I1}, if the
type of \texttt{i1} is annotated as \texttt{D}, the choice of which
function is used is made in module \texttt{I1}, and thus the instance
defined in \texttt{I1} must surely be the instance used.

A Haskell module exports functions with defined types, and a type
annotation can change a defined type. If a module exports a function
with a type such as, for example, \texttt{Num a => a -> a}, the
insertion of a type annotation can change this type, for example to
\texttt{Int -> Int}. A module that imports this function, and uses it
with type \texttt{Integer -> Integer} will not compile, even if the
function definition remains the same.  Thus, a type annotation
included in a top level declaration can change the interface of a
module, and it is reasonable that some programs will then stop
working.  When the interface of a module changes, because of a change
in the type of an exported function, it is reasonable that the
semantics of the exported function can change.

Our proposal makes it possible for a change in type annotations to
cause semantic changes, but only between modules and not inside a
module. Such a semantic change can occur only when the interface of a
module changes, by a change in the type of an exported function. In
the example, function \texttt{i1} with type annotation \texttt{D} is
not, in any way, related to type class \texttt{T}, and should thus not
be affected by instances declared in the importing module.  On the
other hand, if no type is annotated, or a type that has a constraint
on \texttt{T} is annotated, function \texttt{i1} will be related to the
type class, and its use can thus be affected by the definition or
existence of instances of this type class.  Notice that there exist
already other examples of cases of type annotations affecting the
semantics of Haskell programs, related to the use of defaulting
rules\footnote{Described in e-mails
  \url{http://www.haskell.org/pipermail/haskell/2001-May/007113.html}
  ,
  \url{http://www.haskell.org/pipermail/haskell/2001-May/007118.html}
  and
  \url{http://www.haskell.org/pipermail/haskell/2001-May/007117.html}
  .} and an ``a \textit{really\/} amazing example''\footnote{As
  mentioned by Simon Peyton-Jones in
  \url{http://www.haskell.org/pipermail/haskell/2001-May/007133.html}
  .} using polymorphic recursion\footnote{Described by Lennart
  Augustsson in
  \url{http://www.haskell.org/pipermail/haskell/2001-May/007122.html}
  .}.  We believe that the advantages of our proposal outweigh
disadvantages related to these issues.

\chapter[Extending the Module System specification]{Extending Haskell's Module System Formal specification}
\label{formal}
The module system of Haskell 98 has been formally specified
\citep{formal} without dealing with type class instances. This chapter
presents an extension of this formalization for dealing with type
class instances, including the changes needed in \citep{formal} in
order to cope with both the intermediate and final alternatives of our
proposal.  The work in which the formalization is made does not
provide the complete code of the formalization, but the code is
available on the
web\footnote{\url{http://yav.purely-functional.net/publications/modules98-src-21-Nov-2005.tar.gz}.}.

The code models \texttt{Name} as a wrapper around a \texttt{String},
and it is stated in the work that type class instances were not
considered because it is not possible to refer to them by a
name \citep[section~3.1]{formal}. We propose that names of instances be
written as they occur in export and import clauses (as presented in
Figures \ref{export} and \ref{import}).  By doing this, there is no
need to change data type \texttt{Name}, nor data type \texttt{Entity}
used for describing exported and imported entities.

\begin{figure}
\caption{Auxiliary functions for filtering instances in the module
  system.\label{new}}
\begin{tabular}{|p{\textwidth}|}
\hline
\begin{verbatim}
isInst :: Entity -> Bool
isInst (Entity { name = n }) = head (words n) == "instance"
isInst _ = False

instances :: (Ord a) => Rel a Entity -> Rel a Entity
instances = restrictRgn isInst
\end{verbatim}
\\
\hline
\end{tabular}
\end{figure}

For the Instance Names extension, presented in Section
\ref{Instance-names}, instance names can also be used to refer to an
instance. In this case, the name mentioned in the \texttt{Entity} data
type must be the real name of the instance, and not the synonym.
Otherwise, it will not be possible to tell if the name refers to an
instance or not: the auxiliary function \texttt{isInst}, defined in
Figure \ref{new}, is used to distinguish type class instances from
other entities. Funcion \texttt{isInst} is used in the same manner as
function \texttt{isCon}, defined in the paper \citep[section
  3.1]{formal}. Another auxiliary function that should be defined is a
filter for type class instances, called, say, \texttt{instances} (see
Figure \ref{new}), to be used for the changes introduced in our
extension of the formalization. 
% Function \ref{restrictRng} .... 

\section{Haskell and the intermediate alternative}
Our proposal can be applied to both Haskell 98 or Haskell 2010, since
the language changes from Haskell 98 to Haskell 2010 do not affect the
proposal.  The changes needed to be done in the formalization of the module
system for including the way Haskell deals with type class
instances and the way our intermediate proposal deals with it are the
same.  The difference is that our proposal provides some syntatic
constructs which are not available in Haskell.  From the
perspective of the module system specification, this will mean that
some possibilities, like hiding an instance, are not going to happen,
but having the code for it available will not interfere with the
result.  Because of this, in this subsection we present the changes
needed for both Haskell and our intermediate proposal.

Only two things need to be changed in the specification: the way
exported and imported entities are obtained.  In the case of exported
entities, function \texttt{exports} \citep[section~5.2]{formal} needs
to be changed. The old version of the function is presented in Figure
\ref{old-exports} and the new version in Figure \ref{new-exports}.
The difference between them is just that, when a export list is
available (the \texttt{Just es} case) the instances are exported with
what is on the export list.  The instances, then, are always exported,
as defined in Haskell 2010 report \citep[section 5.4]{report}.
%Where/how/when instances are inserted in the export list?

\begin{figure}
\caption{Function \texttt{exports} as in \citep[section 5.2]{formal}.\label{old-exports}}
\begin{tabular}{|p{\textwidth}|}
\hline
\begin{verbatim}
exports :: Module -> Rel QName Entity -> Rel Name Entity
exports mod inscp =
  case modExpList mod of
    Nothing -> modDefines mod
    Just es -> getQualified `mapDom` unionRels exps
      where exps = mExpListEntry inscp `map` es
\end{verbatim}
\\
\hline
\end{tabular}
\end{figure}

\begin{figure}
\caption{New function \texttt{exports}.\label{new-exports}}
\begin{tabular}{|p{\textwidth}|}
\hline
\begin{verbatim}
exports :: Module -> Rel QName Entity -> Rel Name Entity
exports mod inscp =
  case modExpList mod of
    Nothing -> modDefines mod
    Just es -> unionRels
        [getQualified `mapDom` unionRels exps,
          instances $ modDefines mod_]
      where exps = mExpListEntry inscp `map` es
\end{verbatim}
\\
\hline
\end{tabular}
\end{figure}

The other change needed, which is related to imported entities, is on
function \texttt{mImp}.  The change deals with a function defined in
the \texttt{where} clause of function \texttt{incoming}.  The old and
new versions of function incoming are presented respectively in
Figures \ref{old-incoming} and \ref{new-incoming}.  Similarly to the
change in the \texttt{exports} function, this change includes
instances in entities that are going to be imported even if they are
not in the import list.

\begin{figure}
\caption{The function \texttt{incoming} as it is on \citep[section
    5.3]{formal}, for reference.\label{old-incoming}}
\begin{tabular}{|p{\textwidth}|}
\hline
\begin{verbatim}
incoming
  | isHiding = exps `minusRel` listed
  | otherwise = listed
\end{verbatim}
\\
\hline
\end{tabular}
\end{figure}

\begin{figure}
\caption{The new \texttt{incoming} function that also deals with
  instances.\label{new-incoming}}
\begin{tabular}{|p{\textwidth}|}
\hline
\begin{verbatim}
incoming
  | isHiding = exps `minusRel` listed
  | otherwise = unionRels [listed, instances exps]
\end{verbatim}
\\
\hline
\end{tabular}
\end{figure}

Notice that, in the case of a hiding import such that an instance is
on the hiding list, in the intermediate alternative the instance will
not be imported, as expected, because instances are only being added
in the case where they are not a hiding import.  Also, if the instance
is not on the hiding list, it will be imported, because it is included
in \texttt{exps}.

\section{The final alternative}
To specify the final alternative, the consideration about how to use the
instances as names is still valid, in order to allow the system to recognize
instances, but the
rest of the specification must be kept in the same way as it is, that is, without
the changes proposed in the last subsection.  This happens because our proposal
makes instances be treatable in the same fashion as other Haskell entities, so
that the specification that worked for them works also for instances.

\chapter{Related work}
\label{related}
The work of Named instances \citep{named} solves issues related to those
discussed in our work.  In that work a new name must be given for each instance,
and the name must be used to reference the defined instance.  This implies big changes to the
language, including ``how much context reduction should be done before
generalization'' \citep[p.~8]{tc}.  Our proposal is simpler, since it requires fewer
changes in the language and is, therefore, more likely to be included and
internalized by Haskell programmers.

Named instances provide more expressivity than our proposal, because it allows
any two different instances of the same type class for the same data type to be
used in the same module.  In our proposal, two different instances of the same
type class for the same data type can only be used in two different modules.
This can be a problem because our proposal forces the programmer to split a
module in two in this situation, but we do not believe that the need to
write more than one instance per type class and data type will be common. The
burden of creating a new module is, then, not very severe.  Thus, while we lose on expressivity,
we gain on simplicity and we think that this is a good trade-off.

Another related work is that on \emph{scoped instances} \citep{scoped}, which
suggests a language extension for Haskell that allows instances to be
defined inside \texttt{let} clauses. An example is given in Figure \ref{scoped}. The
proposal suggests choosing the instance that is in the innermost scope,
allowing in this scheme also overlapping instances. The proposal does not
deal though with the problems of visibility of instances across modules, and
thus does not solve the problems of orphan instances nor the problem of
pollution of module scopes.

\begin{figure}
\caption{Example of scoped instance extracted from \citep[section~6]{scoped}.\label{scoped}}
\begin{tabular}{|p{\textwidth}|}
\hline
\begin{verbatim}
e2 = let instance Eq Int where
           x == y = primEqInt (x `mod` 2) (y `mod` 2)
     in 3 == 5
\end{verbatim}
\\
\hline
\end{tabular}
\end{figure}

Dreyer, Harper, Chakravarty and Keller have proposed a more radical
change to Haskell that allows ``viewing type classes as a particular
mode of use of modules'' \citep{modular}. Their work also identifies
drawbacks of the current state of the Haskell's type class mechanism
--- namely, lack of modularity, with consequent inconveniences for the
programmer of having always only one instance of a type class for any
type, and lack of separation from definition of instances to their
availability of use. They also identify a problem of coherence, namely
that semantics might differ based on a decision of overloading
resolution made by the type inference algorithm. Their solution is to
require that the scope of instances be confined to the global module
level, where required type annotations identify whether overloading
has been resolved and, if not, the set of permissible instances. In
our proposal, as in Haskell, instances are always at the global module
level (our proposal simply allows control of which instances are
imported and exported). Overloading resolution is based on the type of
the exported instance. If overloading is not resolved, the set of
permissible instances is the set of available instances in the
importing module.

\chapter{Conclusion}
\label{conclusion}

The Haskell language extension proposed in this work gives more
freedom to programmers. On the negative side, this can lead to misuses
that may cause programs to become harder to read and to reason about,
because assumptions about, for example, the behavior of functions like
\texttt{sort} may not hold if a non-standard instance of class
\texttt{Ord} is used.
Also, certain operations rely on the presence of some instances, and
programmers must be aware of that when redefining instances.  Finally,
the inclusion of type signatures can change the semantics of a program
if such type signatures cause types of exported functions, and
instance selection, to be modified. Programmers must then be aware of
that and be careful when changing the type of exported entities.

On the positive side, our proposal makes only small changes to the
language syntax and semantics. It gives more control to programmers
which may construct now programs and libraries that are simpler and
more readable.  The proposal removes the necessity of the
\texttt{...By} class of functions and well-known and often discussed
problems related with orphan instances. The proposal also makes
exportation and importation of instances more homogeneous with other
entities, as shown by the fact that the formalization does not need to
be changed to deal with instances in our final proposal, but it does
need to be changed to handle instances as they are in Haskell
nowadays.

\section{Future work}
This work has presented both syntactic and semantic details of our
proposal. An implementation of both syntax alternatives, specifically in
the most used Haskell compiler GHC, still needs to be done. The
inclusion of a good quality implementation in the main distribution of
GHC will allow programmers an opportunity to use the extension on
production code, enabling a good evaluation of the utility of the
extension in the real world.  Rafael Alcântara de Paula is working on
implementing this proposal in
a Haskell compiler prototype, developed by Rodrigo Ribeiro. The source
code of this compiler is available at
\url{https://github.com/rodrigogribeiro/core}.

\ppgccbibliography{marcot}

\end{document}
