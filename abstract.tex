The Haskell module system aims for simplicity and has a notable
advantage of being easy to learn and use. However, type class
instances in Haskell are always exported and imported between
modules. This breaches uniformity and simplicity of the module system
and introduces practical problems.  Instances created in different modules can
conflict with each other, and can make it impossible to import two
modules that contain the same instance definitions if this instance is used.  Because of
this, it is very incovenient to define two distinct instances of the same type class
for the same type in a program.  The
definition of instances in modules where neither the data type nor the type class are
defined, called orphan instances, became a bad practice.  Only
these instances can cause conflicts since, if instances are
defined in the same module of the type or of the type class, only
one instance can possibly exist for each pair of class and type.

In this dissertation we present and discuss a solution to these problems that
simply allows control over importation and exportation of instances between
modules, through a small change in the language.  The solution is
presented in two versions.  The final version, more consistent, is not
compatible with Haskell, that is, Haskell programs may not work
with this change.  The intermediate version, on the other hand, brings the
benefits of the proposal while being compatible with Haskell, but it is
less consistent.  In order to avoid very long
names for instances in module importation and exportation control lists, we propose another small change in the language to make it possible
to give shorter names to instances.

We also show how a formal specification of the module system must be
adapted to include our proposal.  As the formal specification didn't
handle instances in general, we first adapt this specification to handle instances, and then show how our proposal can be formally specified.

\keywords{Type class instances, Modules, Haskell}
